\documentclass[a4paper]{article}
\usepackage{booktabs,amsmath,varwidth}
\title{COMP6262 Logic Assignment - 3}
\author{Mukund Balaji Srinivas - u7274095}
\begin{document}
\maketitle
%%Introduction to the problem
\section*{Introduction}
Unlike the implication used in = natural language that can indicate causation, formal logic on the other hand is un-intuitive. Consider an example the phrase \textit{philosophers are superhuman } can have two different meanings. If one is a philosopher, one is superhuman or all superhuman are philosophers. \linebreak  
To circumvent these ambiguities, formal logic defines a truth table as below : 
\begin{center}
\begin{tabular}{|c|c|c|}
	$p$ & $q$ & $p \rightarrow q$ \\
	\midrule
	$1$ & $1$ & $1$ \\
	$1$ & $0$ & $0$ \\
	$0$ & $1$ & $1$ \\
	$0$ & $0$ & $1$ \\	
\end{tabular}
\end{center}
However, such strict interpretations are often at odds with the way implication is perceived in natural language. and thus are paradoxical\\
This discussion shall be centered around the following implications:
\begin{enumerate}
	\item{$A \vdash (A \rightarrow B)$}
	\item{$\lnot B \vdash (A \rightarrow B)$}	
\end{enumerate}
Following which a new paradigm for solving implication shall be introduced that circumvent the demerits of the first definition

\section*{Solution to the paradox}
The objective of introducing a new paradigm to precisely and coherently formalize the implication used in natural language would be :
\begin{center}
	\begin{tabular}{|c|c|c|}
		$p$ & $q$ & $p \rightarrow q$ \\
		\midrule
		$1$ & $1$ & $1$ \\
		$1$ & $0$ & $0$ \\
		$0$ & $1$ & $0$ \\
		$0$ & $0$ & $1$ \\	
	\end{tabular}
\end{center}
\pagebreak
The merits and demerits along with the proofs of sequents [1] and [2] from the introduction should be discussed by proving them using both the old and new natural deduction rules that need to be introduced.\\
%% Elimination rules
\begin{center}
		\begin{tabular}{c c}
			 Modified Implication Elimination ... (1)\\
			\midrule 
			 $X\vdash A \rightarrow B $ & $ Y\vdash \lnot B$  \\ 	\midrule
			  \hspace{5em} $X,Y \vdash \lnot B$
		% 		
	\end{tabular}
\end{center}

\begin{center}
	\begin{tabular}{c c}
		Modified Implication Elimination ... (2)\\
		\midrule 
		$X\vdash A \rightarrow B $ & $ Y\vdash A$  \\ 	\midrule
		\hspace{5em} $X,Y \vdash  B$
		% 
		
		
	\end{tabular}
\end{center}

%% Introduction rules
\begin{center}
	\begin{tabular}{c c}
		Modified Implication Introduction ... (3)\\
		\midrule 
		$X,A\vdash B $  \\ 	\midrule
		\hspace{5em} $X\vdash A\rightarrow B$
		% 		
	\end{tabular}
\end{center}

\begin{center}
	\begin{tabular}{c c}
		Modified Implication Introduction ... (4)\\
		\midrule 
		$X,\lnot A \vdash \lnot B $  \\ \midrule
		\hspace{5em} $X \vdash \lnot A \rightarrow \lnot B$
		% 
		
		
	\end{tabular}
\end{center}


	

\end{document}